% 01.tcp.file.transfer.rpc.tex
\documentclass[a4paper,11pt]{article}

\usepackage[utf8]{inputenc}
\usepackage[T1]{fontenc}
\usepackage{geometry}
\usepackage{graphicx}
\usepackage{listings}
\usepackage{caption}
\usepackage{hyperref}

\geometry{margin=2.5cm}

\lstdefinestyle{PythonStyle}{
  language=Python,
  basicstyle=\ttfamily\small,
  keywordstyle=\bfseries,
  numbers=left,
  numberstyle=\tiny,
  stepnumber=1,
  showstringspaces=false,
  tabsize=4,
  breaklines=true
}

\title{RPC-based 1-1 File Transfer over TCP/IP in CLI}
\author{Tran Anh Quoc -- 22BA13266\\Class: 22-Cybersecurity}
\date{} % no date

\begin{document}

\maketitle

\section{Introduction}

The original requirement was to build a simple one-to-one file transfer
system over TCP/IP using a client--server architecture and the classic
socket API (\texttt{socket}, \texttt{bind}, \texttt{listen},
\texttt{accept}, \texttt{connect}, \texttt{send}, \texttt{recv}, etc.).
In the first version, a custom application-layer protocol was defined on
top of a raw TCP stream.

In this report, the system is upgraded to use \textbf{RPC (Remote
Procedure Call)} instead of manually sending and parsing headers over
the TCP connection.  The concrete choice of RPC service is
\textbf{XML-RPC in Python}, implemented using the standard library
modules \texttt{xmlrpc.server} and \texttt{xmlrpc.client}.  Under the
hood, XML-RPC still relies on TCP sockets and HTTP, but it exposes a
clean function-style interface to the application.

The final result is a CLI-based system where the client simply calls a
remote function:

\begin{verbatim}
upload_file(filename, binary_data) -> bool
\end{verbatim}

and the RPC framework takes care of marshalling, transport and
unmarshalling.

\section{Baseline TCP Design (Summary)}

The baseline design before upgrading to RPC used a single TCP connection
and a custom header:

\begin{itemize}
  \item 1 byte: opcode (0x01 = send file)
  \item 2 bytes: file name length (unsigned, big-endian)
  \item 8 bytes: file size in bytes (unsigned, big-endian)
  \item $N$ bytes: file name (UTF-8)
  \item file content: exactly ``file size'' bytes
\end{itemize}

The server used \texttt{socket()}, \texttt{bind()}, \texttt{listen()}
and \texttt{accept()} to establish the connection, then repeatedly
called \texttt{recv()} to read the header and file content.  The client
used \texttt{gethostbyname()} and \texttt{connect()} to reach the
server, then \texttt{send()} to push header and data.

This design already achieved reliable file transfer using TCP, but the
application code was tightly coupled to low-level details: header
packing/unpacking, manual loops to ensure all bytes were sent and
received, etc.  RPC abstracts these details away.

\section{RPC-Based Protocol Design}

\subsection*{RPC interface}

In the upgraded system, the application-level protocol is expressed in
terms of a single remote procedure:

\begin{verbatim}
bool upload_file(string filename, binary data)
\end{verbatim}

The semantics are:

\begin{itemize}
  \item Input:
    \begin{itemize}
      \item \texttt{filename}: name of the file to create on the server.
      \item \texttt{data}: raw file content as a byte sequence.
    \end{itemize}
  \item Output: \texttt{True} if the file was stored successfully,
        \texttt{False} otherwise.
\end{itemize}

The XML-RPC framework automatically:

\begin{itemize}
  \item serialises the function name and arguments into XML,
  \item sends them in an HTTP request over TCP to the server,
  \item calls the corresponding Python function on the server,
  \item serialises the return value and sends it back to the client.
\end{itemize}

From the application's point of view, calling a remote procedure looks
almost the same as calling a local function.

\subsection*{Protocol figure}

\begin{figure}[h]
  \centering
  \fbox{\parbox{0.92\linewidth}{
   \textbf{Client} \hfill \textbf{RPC Server}\\[0.4em]
   1. Application opens local file and reads its content.\\[0.2em]
   2. Client creates XML-RPC proxy to the server URL.\\[0.2em]
   3. \textbf{Client} $\rightarrow$ \textbf{Server}:\\
   \hspace*{1em}XML-RPC request over HTTP/TCP\\
   \hspace*{1em}Method: \texttt{upload\_file}\\
   \hspace*{1em}Parameters: \texttt{filename}, \texttt{binary data}\\[0.2em]
   4. RPC server unmarshals request and calls local\\
   \hspace*{1em}\texttt{upload\_file(filename, file\_data)}.\\[0.2em]
   5. Server writes data to file on disk.\\[0.2em]
   6. \textbf{Server} $\rightarrow$ \textbf{Client}:\\
   \hspace*{1em}XML-RPC response (\texttt{True}/\texttt{False}).\\[0.2em]
   7. Client checks return value and prints result.\\
  }}
  \caption{RPC-based protocol for one-to-one file upload.}
  \label{fig:rpc_protocol}
\end{figure}

In contrast to the previous custom header design, there is no explicit
opcode or length field in the application code: these details are
handled by the RPC framework.

\section{System Organisation}

The system is organised into two Python scripts:

\begin{itemize}
  \item \textbf{\texttt{rpc\_server.py}}: exposes the remote procedure
        \texttt{upload\_file} via XML-RPC.
  \item \textbf{\texttt{rpc\_client.py}}: connects to the XML-RPC
        service and calls \texttt{upload\_file} to upload a local file.
\end{itemize}

The CLI usage is:

\begin{verbatim}
# On the server machine:
$ python3 rpc_server.py <port>

# On the client machine:
$ python3 rpc_client.py <server_host> <port> <file_path>
\end{verbatim}

\begin{figure}[h]
  \centering
  \fbox{\parbox{0.92\linewidth}{
   \textbf{rpc\_server.py}\\
   - create XML-RPC server bound to 0.0.0.0:\emph{port}\\
   - register function \texttt{upload\_file}\\
   - enter \texttt{serve\_forever()} loop\\[0.6em]
   \textbf{rpc\_client.py}\\
   - resolve server host and build XML-RPC URL\\
   - open local file and read bytes\\
   - call \texttt{upload\_file(filename, Binary(data))}\\
   - print server response and exit
  }}
  \caption{High-level organisation of the RPC-based file transfer system.}
  \label{fig:system}
\end{figure}

Internally, the XML-RPC server still uses the underlying socket API:
\texttt{socket()}, \texttt{bind()}, \texttt{listen()}, \texttt{accept()}
for incoming connections, and \texttt{send()}/\texttt{recv()} for
transport.  However, this low-level logic is encapsulated in the
library instead of being written manually in the application code.

\section{Implementation: File Transfer via XML-RPC}

\subsection*{Server-side implementation}

The RPC server exposes a single function \texttt{upload\_file} that
accepts a filename and a binary blob and writes it to disk.  The
\texttt{Binary} wrapper type from \texttt{xmlrpc.client} is used on the
client side to indicate that the parameter is raw bytes; on the server
side, the incoming object has a \texttt{.data} attribute containing the
actual byte string.

\begin{lstlisting}[style=PythonStyle,
  caption={RPC server exposing the \texttt{upload\_file} service.},
  label={lst:rpc_server}]
#!/usr/bin/env python3
from xmlrpc.server import SimpleXMLRPCServer, SimpleXMLRPCRequestHandler
import sys
import os

class RequestHandler(SimpleXMLRPCRequestHandler):
    # Restrict XML-RPC path for security
    rpc_paths = ("/RPC2",)

def upload_file(filename, file_data):
    """
    Remote procedure:
      - filename: string
      - file_data: object with attribute .data (bytes)
    """
    safe_name = os.path.basename(filename)  # avoid path traversal
    data_bytes = file_data.data

    print(f"[+] Saving file {safe_name} ({len(data_bytes)} bytes)")
    with open(safe_name, "wb") as f:
        f.write(data_bytes)
    print("[+] Done.")
    return True  # status for the client

def main():
    if len(sys.argv) != 2:
        print(f"Usage: {sys.argv[0]} <port>")
        sys.exit(1)

    port = int(sys.argv[1])

    # Create XML-RPC server on 0.0.0.0:port
    with SimpleXMLRPCServer(("0.0.0.0", port),
                            requestHandler=RequestHandler,
                            allow_none=True) as server:
        print(f"[+] XML-RPC server listening on 0.0.0.0:{port}")

        # Register the remote function
        server.register_function(upload_file, "upload_file")

        # Start serving requests
        server.serve_forever()

if __name__ == "__main__":
    main()
\end{lstlisting}

\subsection*{Client-side implementation}

The client reads a local file into memory and sends it to the server by
calling the remote procedure.  The only network operation visible in
the application code is the method call on the XML-RPC proxy object.

\begin{lstlisting}[style=PythonStyle,
  caption={RPC client calling \texttt{upload\_file} to transfer a file.},
  label={lst:rpc_client}]
#!/usr/bin/env python3
from xmlrpc.client import ServerProxy, Binary
import sys
import os

def main():
    if len(sys.argv) != 4:
        print(f"Usage: {sys.argv[0]} <server_host> <port> <file_path>")
        sys.exit(1)

    server_host = sys.argv[1]
    port        = int(sys.argv[2])
    file_path   = sys.argv[3]

    if not os.path.isfile(file_path):
        print("[-] File not found:", file_path)
        sys.exit(1)

    # Build XML-RPC endpoint URL
    url = f"http://{server_host}:{port}/RPC2"
    server = ServerProxy(url, allow_none=True)

    # Read file into memory
    with open(file_path, "rb") as f:
        data = f.read()

    filename = os.path.basename(file_path)
    print(f"[+] Uploading {filename} ({len(data)} bytes) to {url}")

    # Binary(...) tells XML-RPC that this parameter is raw bytes
    ok = server.upload_file(filename, Binary(data))

    if ok:
        print("[+] File uploaded successfully via RPC.")
    else:
        print("[-] Server reported failure.")

if __name__ == "__main__":
    main()
\end{lstlisting}

\subsection*{Mapping to classic socket functions}

Even though the code above does not call the low-level functions
explicitly, the XML-RPC library internally relies on the same primitives
specified in the assignment:

\begin{itemize}
  \item \textbf{On the server side}:
    \begin{itemize}
      \item \texttt{socket()} to create a listening socket.
      \item \texttt{setsockopt()} to configure options such as
            \texttt{SO\_REUSEADDR}.
      \item \texttt{bind()} to associate the socket with an address.
      \item \texttt{listen()} to wait for incoming connections.
      \item \texttt{accept()} to accept a client connection.
      \item \texttt{recv()} and \texttt{send()} to read/write HTTP/XML.
    \end{itemize}
  \item \textbf{On the client side}:
    \begin{itemize}
      \item \texttt{gethostbyname()} to resolve the server hostname.
      \item \texttt{socket()} and \texttt{connect()} to open the TCP
            connection.
      \item \texttt{send()} and \texttt{recv()} to implement HTTP/XML-RPC.
    \end{itemize}
\end{itemize}

Thus the upgraded system still satisfies the requirement to use socket
functions, but in a more structured RPC fashion.

\section{Conclusion}

This report presented a complete redesign of a one-to-one file transfer
system from a custom TCP-based protocol to an RPC-based architecture
using XML-RPC in Python.  By exposing a single remote procedure
\texttt{upload\_file}, the application logic becomes simpler, and the
implementation is cleaner: marshalling, transport and unmarshalling are
delegated to the RPC framework.

The system still relies on the standard socket API underneath, but the
application code operates at the level of function calls instead of raw
byte streams.  This approach can be extended to support additional
operations such as downloading files, listing files on the server or
integrating authentication and access control, simply by adding more
remote procedures.

\end{document}
